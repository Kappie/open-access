\documentclass[11pt, openany, oneside, article, a4paper, twocolumn]{memoir}
\usepackage{lipsum}
\usepackage{fontspec}
\usepackage{csquotes}
\usepackage[backend=biber, style=phys, hyperref, url=false, sorting=none]{biblatex}
\usepackage{hyperref}

\addbibresource{references.bib}
%\setmainfont[
  %BoldFont={Chronicle Text G1 Bold},
  %ItalicFont={Chronicle Text G1 Italic},
  %BoldItalicFont={Chronicle Text G1 Bold Italic}
%]{Chronicle Text G1 Roman}
%\setmainfont[
  %BoldFont={Miller Text Bold},
  %ItalicFont={Miller Text Italic}
%]{Miller Text}

\setmainfont[
  BoldFont={Minion Pro Bold},
  ItalicFont={Minion Pro Italic},
  BoldItalicFont={Minion Pro Bold Italic}
]{Minion Pro}
\setsansfont[]{Consolas}

\setlength{\absleftindent}{5em}
\setlength{\absrightindent}{5em}
\renewcommand{\abstracttextfont}{\footnotesize}
\setlength{\columnsep}{24pt}
\setlength{\parskip}{0pt}
\newlength{\postabstractskip}
\setlength{\postabstractskip}{15pt}
\urlstyle{rm}

% \setbeforesecskip{2.5ex plus 1ex minus .2ex}
% \setaftersecskip{2.5ex plus 1ex minus .2ex}

%\newfontfamily\headingfont[
  %ItalicFont={Chronicle Display Italic},
%]{Chronicle Display}
%\newfontfamily\subheadingfont[]{Chronicle Display Light}

\begin{document}
\renewcommand{\thesection}{\Roman{section}.}
\renewcommand{\abstractname}{}
\renewcommand{\absnamepos}{empty}
\pretitle{\begin{center}\begingroup\fontsize{32pt}{28pt}\selectfont}
\posttitle{\par\endgroup\end{center}\vskip 0.5em}

% Make subsubsection headers run-in
\setbeforesubsubsecskip{0ex}
\setsubsubsecindent{\parindent}
\setaftersubsubsecskip{-1ex}

\title{Scientists: support a self-archiving mandate to free the peer-reviewed journal literature}
\author{Geert Kapteijns\thanks{\href{mailto:ghkapteijns@gmail.com}{ghkapteijns@gmail.com}}}
\date{\today}

\twocolumn[
  \maketitle
  \vspace{-15pt}
  \begin{onecolabstract}

    \noindent Universal free online access of scientific journal articles is
    within reach if universities and funders mandate their authors to
    self-archive their refereed manuscripts in an institutional archive (IR),
    upon acceptance in the (subscription) journals of their choice. This form
    of open access (OA), known as \emph{Green}, can be implemented unilaterally by the
    universities and funders at little cost. It should not be confused with
    \emph{Gold} OA, meaning OA through \emph{publishing} directly in an OA
    journal.

    I claim that the Dutch government and the association of universities
    (VSNU), by focussing on Gold prematurely, have made deals that will
    needlessly slow down the provision of access and maintain or even increase
    the publishers' profit margins. Sustainable Gold access (including
    copyright and re-use rights reform) plausibly follows once universal
    Green has been reached and publishers only provide the organisation of
    peer-review and luxury services like enhanced PDFs and paper versions.

    Change must be effected by institutions and funders, but they have to
    be persuaded by scientists to install the optimal self-archiving mandate.
    Grassroots publishing initiatives, such as SciPost in physics,
    politicize the community by offering a glimpse of a possible future.
    In order to speed up the transition to a fair publishing industry, they
    should, in addition to their innovative activities, put their full
    weight behind Green.
    \end{onecolabstract}
    \vspace{\postabstractskip} ]

\saythanks

\section{Introduction}

The current accessibility of research journal articles is decidedly
suboptimal. Journal prices have been rising at 2.5 times the rate of
inflation the last couple of decades \cite{monograph_serial_costs,
suber2008open}, but even if all 28000 existing journals could be
subscribed to at production cost, universities would not be able to afford
them all \cite{harnad2008access}. No researchers, not even those at the
richest institutions, have full access to the output of their colleagues,
and all researchers are denied the full impact of their research, since
they cannot reach the entirety of their intended audience.

It is unbearable that this \emph{access/impact problem} still persists,
because with the advent of the Web, articles can be reproduced and spread
at virtually no cost. Doubly unbearable, since the whole enterprise is
funded with tax-payer money for the benefit of society. 

The solution to the problem is, according to Stevan Harnad, closer to
raincoat science than rocket science (see \autoref{fig:raincoat_science}). In a haiku \cite{harnad_raincoat}:

\pagebreak

\begin{displayquote}
\begin{small}
It's the online age\\
You're losing research impact\dots \\
Make it free online
\end{small}
\end{displayquote}
In other words, authors can continue publishing in subscription journals,
but should as an extra adminstrative step \emph{publicly self-archive}
their refereed manuscripts. This practice, which Harnad has been
advocating since 1994 \cite{harnad1995subversive}, was laid out by the
Budapest Open Access Initiative (BOAI) as the first strategy to be
implemented \cite{boai}. It later came to be known as the \emph{Green}
strategy \cite{harnad2004access}. If it is universally adopted, the
access/impact problem would be solved.

\begin{figure}
  \includegraphics[width=\columnwidth]{rain4.jpg}
  \caption{Raincoat science by Judith Economos \cite{economos}.}\label{fig:raincoat_science}
\end{figure}

Apart from public self-archiving (the \enquote{Green} road), the BOAI
described a complimentary \emph{Gold} strategy, namely to start a new
generation of journals that provide open access to the material they
publish. It is this second strategy that often has been misunderstood to
be the \emph{only} viable strategy of providing open access, by
scientists, media and politicians alike.

This is very unfortunate, since the Green solution is by far the most
cost-effective way of providing access \cite{houghton2013planting}
(10\%-20\% of what it presently costs to pay for Gold), can be decided on
by universities and funders unilaterally, without having to convince
publishers to alter their business model, and does not limit authors'
choices of journals in which they wish to publish.

Furthermore, it is plausible that once universal Green open access has been
achieved, existing subscription journals will face significant cancellation
pressure, because all their content is already available as self-archived
manuscripts. Publishers would be forced to cut costs and change their
business model, since the services they provide have been reduced to organizing
peer-review and providing enhanced PDFs and paper versions of the articles.
Thus, Green OA will leverage the transition to universal Gold
\cite{harnad2007green_road}.

In the rest of this article, I will first outline what is currently
understood to be the optimal Green open access policy. Then, I will show
that official policy in The Netherlands seriously deviates from this
consensus, needlessly slowing down the provision of access and maintaining
or even increasing the publishers' profit margins. Finally, I describe
that, since change is most likely to come from below, it is of vital
importance to the community that grassroots initiatives embrace the Green
mandate.

\section{The optimal Green mandate}

Apart from being beneficial to the community, self-archiving is also
advantageous for researchers personally due to increased uptake and
citation impact of their work \cite{gargouri2010self}.

Yet, the majority
of scientists do not self-archive voluntarily. 62\% of journals endorse
self-archiving immediately and an additional 17\% endorse self-archiving
after an embargo period of six months or a year \cite{bjork2014anatomy}.
But estimates for the actual percentage of articles that is accessible in
this way (be it from an institutional archive, a preprint server or the
author's homepage) are far lower. The authors of \cite{bjork2010open} find
that in 2009 20\% of all journal articles were openly accessible, of which
12\% through self-archiving. In a subsequent study, the same authors find
an unchanged 12\% Green in 2014 \cite{bjork2014anatomy}. The authors of
\cite{khabsa2014number} find 24\% total OA (Green and Gold) in 2013. The
study in \cite{archambault2013tipping} is an outlier, finding 48\% total
OA, of which 34\% Green\footnote{The authors also include articles
published in \emph{hybrid} journals -- meaning subscription journals that
offer the option of providing open access for an additional author fee --
in this percentage, so the fraction of Green articles is presumably
slightly lower.}, already in 2008.

These numbers include unrefereed preprint versions (about 15\% in
\cite{bjork2014anatomy}), since the archived and published versions are
mostly matched by automated title/author/abstract matching. Furthermore,
archived manuscripts are scattered throughout the Web
\cite{kim2010faculty}, and archived versions that became available only
after a (possibly long) embargo period are counted.

In the domain of Physics and Astronomy, where sharing preprints has
historically played an important role \cite{brown2001evolution},
self-archiving is universally endorsed by publishers\footnote{This is an
example of \enquote{proof by intimidation.} Feel free to prove me wrong
using \cite{sherpa_romeo}. In any case, physicists, along with computer
scientists and mathematicians, have always freely shared their preprints
and refereed manuscripts. E-mail and later preprint servers became natural
tools to make this practice easier and were freely used, even before the
issue of self-archivation ended up in publisher contracts
\cite{brown2001evolution}. Publishers have, to the best of my knowledge,
never ordered anyone to take down a manuscript in these fields.}. The
preprint server arXiv, established in 1991, has become the canonical place
to share manuscripts. But even in this field, self-archiving is not
systematic, although the numbers are slightly higher. Estimates are that
around 20\% of papers that appear in Web of Science journals can be found
on arXiv, possibly as an unrefereed (preprint) version
\cite{bjork2010open, lariviere2014arxiv}. Some subfields, such as
astronomy and high-energy physics, have around 70\% Green with the
percentage in top journals approaching 100\% \cite{gentil2010citinghep}.

If the benefits are clear, why do scientists refuse to self-archive?
Harnad lists many possible reasons \cite{harnad2006opening}, the most
prevalent being that (i) scientists think it is illegal, (ii) that it
causes their papers to be less likely to be accepted, and (iii) that
scientists are simply too lazy.

The solution is for universities and funders to \emph{mandate} their
researchers to self-archive. It is worth quoting Harnad's implementation proposal in full \cite{harnad2015openwhat}:
\begin{small}
\begin{enumerate}[(1)]
\item All research funding agency OA Mandates need to specify clearly and explicitly that the deposit of each
article must be in the author’s institutional repository (so the universities
and research institutions can monitor their own output and ensure compliance as
well as adopt mandates of their own for their unfunded research output).
\item All mandates should specify that the deposit (of the authors refereed, revised,
accepted final draft) must be done immediately upon acceptance for publication
(not on the date of publication, which is often much later, variable, not known
to the author, and frequently does not even correspond to the journal issue’s
published date of publication, if there is one).
\item All mandates should urge
(but not require) authors to make their immediate-deposit immediately-OA.
\item All mandates should urge (but not require) authors to reserve the right to make
their papers immediately-OA (and other re-use rights) in their contracts with
their publishers (as in the Harvard-style mandates).
\item All mandates should
shorten (or, better, not even mention) allowable OA embargoes (so as not to
encourage publishers to adopt them).
\item All repositories should implement the
automated "email eprint request" Button (for embargoed [non-OA] deposits).
\item All mandates should designate repository deposit as the sole mechanism for
submitting publications for performance review, research assessment, grant
application, or grant renewal.
\item All repositories should implement rich usage
and citation metrics in the institutional repositories as incentive for
compliance.
\end{enumerate}
\end{small}

A few of the points are worth stressing. Articles accepted in the 38\%
\cite{bjork2014anatomy} of journals that impose embargoes on
self-archiving or do not allow it at all should still be deposited (point
3).
The institutional repository should implement an \enquote{email eprint
request} or \enquote{fair dealing} button \cite{sale2010open} (point 6) that allows
prospective readers of these deposited non-OA articles to request an individual
copy with a single click, which the author can acknowledge with
another single click. This is completely legal, except under almost
inconcievable circumstances, since the articles are shared for the purpose of
\enquote{study, criticism, and news reporting.}\footnote{The unconvinced reader may
read section \enquote{Legal and policy considerations} from \cite{sale2010open}
or consult her local legislation.} The institutional repository merely
facilitates the age-old scientific practice of handing out personal-use copies
to interested colleagues, thus providing \enquote{almost-OA.}

Equally important is what is \emph{not} in the mandate. Harnad:
\begin{displayquote}
\begin{small}
[It is essential] not to insist prematurely on further rights
-- over and above free online access -- that publishers are not yet willing to
allow, such as text-mining, re-mix and re-publication rights. First things
first: Funders, institutions and authors should not prolong their failure to
grasp what's already within their reach by over-reaching for what's not yet
within reach: The perfect should not be allowed to become the enemy of the
good.
\end{small}
\end{displayquote}

An optimal self-archiving mandate, if universally adopted by universities and insitutions, solves
the access/impact problem without forcing publishers to change
their business model, without limiting authors in their choice of journals and at low cost.
Once the access and archivation of research output is firmly controlled by the community
and the role of publishers is reduced to organising peer-review and
offering luxury products, publishers will have lost their bargaining power
over copyright, which will soon follow.

\section{Current policy in the Netherlands: Fool's Gold}

The current official policy in the Netherlands is about as far from the optimal
solution as one can get. The Dutch secretary of Education, Culture and Science
Sander Dekker, following the Finch report \cite{finch2012accessibility} before
him, outlined the government's stance in his letter to the House of
Representatives \cite{dekker_kamerbrief} in 2013 (citations from English
translation \cite{dekker_kamerbrief_english}):
\begin{displayquote}
\begin{small}
  My preference is “golden” open access; [...] The universities, the Royal
  Academy and NWO [Dutch governmental organisation for scientific
  research] will have to prioritise the golden road to open access in their
  institutional policies [...] Publications will be made publicly available in
  open access journals. Until the publishers have switched to the golden road
  to open access, I prefer a system of hybrid journals in which institutions
  pay to have papers published open access in subscription-based journals.
  Those disciplines in which there are few opportunities to publish in open
  access journals can opt for the green road to open access, in other words by
  having authors self-archive their articles in a repository.
\end{small}
\end{displayquote}

The letter urges universities and funding organisations to encourage their
authors to publish directly in OA journals. Self-archiving is seen as
a final resort for disciplines in which those journals are not (yet)
available. Embargoes on self-archiving are presented as laws of nature, fundamental rights
of publishers and the
possibility of a \enquote{fair use button} is not mentioned.
Dekker threatened to mandate the
Gold strategy by law if deals between publishers and universities did not follow
fast enough:
\begin{displayquote}
\begin{small}
If the relevant parties do not do enough, or progress is unacceptably slow, the
Minister and I will recommend making open access publication mandatory in 2016
under the Higher Education and Research Act (Wet op het hoger onderwijs en
wetenschappelijk onderzoek, WHW).
\end{small}
\end{displayquote}

And indeed, the very timid VSNU (Vereniging van Samenwerkende Nederlandse
Universiteiten or the association of Dutch universities) made deals with
all major publishers in the years that followed \cite{timeline_oa},
concluding in their self-congratulatory report \cite{ezine_oa} that \enquote{The Netherlands became a guiding country for open
access} by \enquote{using Big Deals as a crowbar,} while designating Green as unsustainable.

The deals \cite{uitgeversdeals} only provide 20\% - 30\% open access in the short-term, mostly in
existing subscription journals, paid for by the Dutch universities
(who also still have to pay subscription fees for incoming content). 
They allow publishers to keep charging
extravagant prices for services that could be in the hands of research
community, while smoothly transitioning into the \enquote{author
processing fees} business model.
Moreover, the deals do not scale and are unstable
\cite{harnad2016evolutionary}. Harnad:
\begin{displayquote}
\begin{small}
The notion of a \enquote{flip} to fool's gold is incoherent -- an
\enquote{evolutionary unstable strategy,} bound to undo itself: not only because it
requires self-sacrificial double-payment locally as well as unrealistic
collaboration among nations, institutions, funders, fields and publishers
globally, but because the day after it was miraculously (and hypothetically)
attained globally it would immediately invite defection (from nations,
institutions, funders, and fields) to save money (invasion by the \enquote{cheater
strategy}). Subscriptions and gold OA \enquote{memberships} are simply incommensurable,
let alone transformable from one into the other. (Memberships are absurd, and
only sell -- a bit, locally -- while subscriptions still prevail, via local Big
Deals.)
\end{small}
\end{displayquote}

Although every university is equipped with a repository \cite{narcis} and the
universities, libraries and funding organsiations are in favor of Green (see
\cite{openaccessnl} -- TU Eindhoven even has a mandate that is reasonably
good), the Dutch government and VSNU have bought \enquote{Fool's Gold,} showing
themselves to be firmly on the side of publishers. In an opinion piece directed
at his international colleagues, called \enquote{For Europe's Fifth Freedom},
Sander Dekker drops all pretense that he is not a corporate avatar
\cite{dekker_fifth_freedom}:

\begin{displayquote} 
\begin{small}
Europe has the world’s largest internal market, seeking to guarantee the free
movement of goods, capital, services and people [...] [that have lain] fertile ground for
decades of peace and prosperity. [GK: for whom?] [...] Ministers in Europe
concerned with science and innovation should thus bring a \enquote*{fifth
freedom} into play: the free movement of knowledge and data. [...] In the
corporate world, data is increasingly supplanting oil as the main economic
asset. But the research data from our smartest people too often lies gathering
dust on easily-misplaced USB sticks. The central storage of data and tools to
consult and re-use it, are often lacking. In an era when big data technology
and services are expected to grow at a compound annual rate of 40 per cent,
this is nothing less than full-blown capital destruction.
\end{small}
\end{displayquote}

No surprise that also the \enquote{Amsterdam Call for Action on Open Science}
\cite{amsterdam_call} produced by the Dutch Presidency of the Council of the
European Union heavily emphasizes Open Data. Open Data, re-use rights or author
copyright is premature if there is not even Open Access, and divert scientists
from their most important immediate goal: providing free access to their own
output through self-archiving.

\section{A grassroots movement}

The path that the government has taken is clear (and given the next
imminent right-wing government, not likely to change). The change has to
be made on the level of universities and funders, who must install the
right mandate. It is up to scientists to convince them to do so.

Grassroots open access publishing initiatives (like SciPost \cite{scipost}
in physics) play a large role in politicizing the community: they spread
the message that not only open access, but complete author copyright and
peer-review reform are possible and can be provided at a fraction of the
current cost. But if they are serious about their goals of breaking the
dominance of the publishing industry, they should consider how to best
proceed from the current situation. 

The empirical studies cited in this article are clear. Scientist right now
do not care too much for open access. If they did, they would just make
their works available online (at the very least the ones they legally
could). It is not realistic to expect that the same scientists can be
persuaded to publish en masse in new fair-Gold journals, while they could
also publish in the current top journals in their field. In fact,
scientist-run refereed open access journals with peer-witnessed
commentaries have been around since 1990 \cite{psycoloquy} and they have
not replaced the corporate publishers' journals.

It is thus vital that the grassroots journals of the current generation
use the positive attention they receive to make clear to universities and
funders that \emph{they} can free the journal literature by mandating
their researchers to self-archive. Once this mandate spreads among
institutions and funders worldwide, the cancellation pressure on the
existing journals will leverage the transition to a stable fair-Gold
system.

\subsubsection{Acknowledgements} Thanks to J.-S. Caux and Stevan Harnad
for valuable discussions.

\printbibliography


\end{document}
