\documentclass[12pt,twocolumn, openany, article]{memoir}
\usepackage{lipsum}
\usepackage{fontspec}
\usepackage{csquotes}
% \setmainfont[
%   BoldFont={Chronicle Text G1 Bold},
%   ItalicFont={Chronicle Text G1 Italic},
%   BoldItalicFont={Chronicle Text G1 Bold Italic}
% ]{Chronicle Text G1 Roman}
\setmainfont[
  BoldFont={Miller Text Bold},
  ItalicFont={Miller Text Italic}
]{Miller Text}
\newfontfamily\headingfont[
  ItalicFont={Chronicle Display Black Italic},
]{Chronicle Display Black}
\newfontfamily\subheadingfont[]{Chronicle Display Light}
% \usepackage[cmintegrals,cmbraces]{newtxmath}
% \usepackage{ebgaramond-maths}
% \usepackage[T1]{fontenc}

\begin{document}
\pretitle{\begin{center}\HUGE\headingfont}

\title{Lorem de porem}

\author{Geert Kapteijns}
\date{\today}
\maketitle

\section*{Open access deal with Elsevier}
Why is the deal bad?

\begin{itemize}
  \item the contracts were secret to begin with, which is a most ridiculous state of affairs
  \item why did the UvA not support the open-information request?
  \item Max 3600 articles.
  \item fees are raised
\end{itemize}

who is making this deal?

\begin{itemize}
  \item Deals are with all Dutch universities?
  But who is \emph{making} them?
  In the press and reports, it seems as if all universities get the same deal,
  brokered by VSNU and UKB (Universiteitsbibliotheken en Koninklijke Bibliotheek)
\end{itemize}

German universities:
Elsevier restored access in face of boycott, while terms are being negotiated.
In other words, they were bluffing, and are now trying to quiet down the researchers by giving them access,
while working out a deal with the German consortium.

Elsevier's worst fear, of course, is that scientists wake up.

Self-congratulating tone of the Netherlands in VSNU-report is out of
place. They signed a deal that is a token step, consolidating power with
Elsevier. Why is the discussion framed in this way? What are the filters
which are applied by corporate interest? Why do we not read 'this is
ridiculous and these companies have got to go' in the official documents?

The solution is simple: scientists should stop publishing in journals run
by media conglomerates, stop using these conglomerates' databases to judge
eachother. This is hard. It requires scientists to confront themselves
with their own culture of \enquote{publish or perish}, and how they judge
eachother's work and body of work.

Sander Dekker should not be taken seriously. His 2013 letter is
scandalous.

SciHub is crucial? Though illegal, it provides for an alternative way to
read publications, therefore giving academics more room to boycott.

\section*{Critique of Sander Dekker's letter}
Dekker completely buys into the idea that the profit margins of the publishers are legitimate.
According to Dekker, the transition period to Open Access publishing should be swift,
so that universities will not incur \emph{double costs}, namely those of subscription fees and publishing fees.
\begin{quote}
De overheid moet richting geven, zodat partijen weten waar ze aan toe zijn en onderling afspraken kunnen
maken.
Een te lange transitieperiode brengt onnodige kosten met zich mee,
omdat de wetenschap dan betaalt voor zowel abonnementen als voor publicaties in tijdschriften.
Een duidelijke keuze voor Open Access van publicaties kan het transitieproces de nodige snelheid geven en de
transitieperiode verkorten en daarmee onnodige extra kosten voorkomen.

\textelp{}

De wetenschap in brede zin, dus universiteiten, NWO, KNAW, VSNU,
Vereniging Hogescholen en NFU en de koepels, heeft belang bij Open Access om resultaten te verspreiden maar ook bij het
behouden van toegang tot wetenschappelijke publicaties uit andere landen.
De uitgevers hebben belang bij een goede business case.
Dat kan ook een nieuwe business case zijn die gebaseerd is op in Open Access publiceren.
\end{quote}

Italics his. Dekker misrepresents the reality around Green open access publishing.
He talks about embargos, which are mostly lifted in most contracts (?) and,
again, sees the embargos as a valid business case.

\begin{quote}
  Uitgangspunt is dus Open Access van publicaties via tijdschriften.
  Zolang de uitgevers de omslag naar Open Access Golden road nog niet hebben gemaakt,
  heb ik voorkeur voor hybride Open Access, waarbij de instelling betaalt voor publicatie in een traditioneel
  tijdschrift.
\end{quote}


\section*{Green vs Gold?}
Why does Dekker see Gold as the best alternative?
The beauty of Green open access is, that it's essentially one step from saying:
we're going to run the show ourselves.

Dekker says that in the absence of good Gold OA journals, researchers should take the hybrid route -- meaning they
should pay, at their own expense, for their work to be published OA,
while the university maintains the subscription to the otherwise closed-access journal. The so-called double costs.

Sander Dekker does not focus on copyright of authors? Misreads Finch report?

\end{document}
