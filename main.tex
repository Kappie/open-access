\documentclass[11pt, openany, oneside, article, a4paper, twocolumn]{memoir}
\usepackage{lipsum}
\usepackage{fontspec}
\usepackage{csquotes}
\usepackage[backend=biber, style=phys, hyperref, url=false, sorting=none]{biblatex}
\usepackage{hyperref}

\addbibresource{references.bib}
%\setmainfont[
  %BoldFont={Chronicle Text G1 Bold},
  %ItalicFont={Chronicle Text G1 Italic},
  %BoldItalicFont={Chronicle Text G1 Bold Italic}
%]{Chronicle Text G1 Roman}
%\setmainfont[
  %BoldFont={Miller Text Bold},
  %ItalicFont={Miller Text Italic}
%]{Miller Text}

\setmainfont[
  BoldFont={Minion Pro Bold},
  ItalicFont={Minion Pro Italic},
  BoldItalicFont={Minion Pro Bold Italic}
]{Minion Pro}
\setsansfont[]{Consolas}

\setlength{\absleftindent}{5em}
\setlength{\absrightindent}{5em}
\renewcommand{\abstracttextfont}{\footnotesize}
\setlength{\columnsep}{24pt}
\setlength{\parskip}{0pt}
\newlength{\postabstractskip}
\setlength{\postabstractskip}{15pt}
\urlstyle{rm}

% \setbeforesecskip{2.5ex plus 1ex minus .2ex}
% \setaftersecskip{2.5ex plus 1ex minus .2ex}

%\newfontfamily\headingfont[
  %ItalicFont={Chronicle Display Italic},
%]{Chronicle Display}
%\newfontfamily\subheadingfont[]{Chronicle Display Light}

\begin{document}
\renewcommand{\thesection}{\Roman{section}.}
\renewcommand{\abstractname}{}
\renewcommand{\absnamepos}{empty}
\pretitle{\begin{center}\begingroup\fontsize{32pt}{28pt}\selectfont}
\posttitle{\par\endgroup\end{center}\vskip 0.5em}
%\renewcommand{\booktitlefont}{\small\headingfont}

\title{hoi ik ben een titel}
\author{Geert Kapteijns}
\date{\today}

\twocolumn[
  \maketitle
  \vspace{-15pt}
  \begin{onecolabstract}

    \noindent Universal free online access of scientific journal articles is
    within reach if universities and funders mandate their authors to
    self-archive their refereed manuscripts in an institutional archive (IR),
    upon acceptance in the (subscription) journals of their choice. This form
    of \emph{Green} open access (OA) can be implemented unilaterally by the
    universities and funders at little cost. It should not be confused with
    \emph{Gold} OA, meaning OA through \emph{publishing} directly in an OA
    journal.

    I claim that the Dutch government and the association of universities
    (VSNU), by focussing on Gold prematurely, have made deals that will
    needlessly slow down the provision of access and maintain or even increase
    the publishers' profit margins. Sustainable Gold access (including
    copyright reform) will follow once universal Green has been reached and
    publishers only provide the organisation of peer-review and luxury services
    like enhanced PDFs or paper versions.

    Grassroots publishing initiatives, such as SciPost, politicize the
    community by offering a glimpse of a possible future. But if they are
    serious about subverting the publishing industry, they should, in addition to
    their innovative activities, put their full weight behind the optimal Green
    mandate.
  \end{onecolabstract}
  \vspace{\postabstractskip}
]

\section{Introduction}

The current accessibility of research journal articles is decidedly
suboptimal. Journal prices have been rising at 2.5 times the rate of
inflation the last couple of decades \cite{monograph_serial_costs,
suber2008open}, but even if all 40000 existing journals could be
subscribed to at production cost, universities would not be able to afford
them all \cite{harnad2008access}. This means that all researchers, even at
the richest institutions, do not have full access to the output of their
colleagues, and all researchers are denied the full impact of their
research, since they cannot reach the entirety of their intended audience.

It is unbearable that this \emph{access/impact problem} still persists,
because with the advent of the Web, articles can be reproduced and spread
at virtually no cost. Doubly unbearable, since the whole enterprise is
funded with tax-payer money for the benefit of society. 

The solution to the problem is, according to Stevan Harnad, closer to
raincoat science than rocket science \cite{harnad_raincoat}.

\blockquote{%
It's the online age\\
You're losing research impact\dots \\
Make it free online}

\begin{figure}
  \includegraphics[width=\columnwidth]{rain4.jpg}
  \caption{Raincoat science by Judith Economos}
\end{figure}

In other words, authors can continue publishing in subscription journals,
but should as an extra adminstrative step \emph{publicly self-archive}
their refereed manuscripts. This practice, which Harnad has been
advocating since 1994 \cite{harnad1995subversive}, was laid out by the
Budapest Open Access Initiative (BOAI) as the first strategy to be
implemented \cite{boai}. It later came to be known as the \emph{Green}
strategy \cite{harnad2004access}. If it is universally adopted, the
access/impact problem would be solved.

Apart from public self-archiving (the \enquote{Green} road), the BOAI
described a complimentary \emph{Gold} strategy, namely to start a new
generation of journals that provide open access to the material they
publish. It is this second strategy that often has been misunderstood to
be the \emph{only} viable strategy of providing open access, by
scientists, media and politicians alike.

This is very unfortunate, since the Green solution is by far the most
cost-effective way of providing access \cite{houghton2013planting}, can be
decided on by universities and funders unilaterally, without having to
convince publishers to alter their business model, and it does not limit
authors' choices of journals in which they wish to publish.

Furthermore, it is plausible that once universal Green open access has
been achieved, 

\section{Raincoat science: the optimal open access policy}

Another solution










\section{To cite}
\begin{itemize}
  \item self-selected or mandated: oa increases impact \cite{gargouri2010self}
  \item green is most cost-effective strategy \cite{houghton2013planting}
\end{itemize}

\twocolumn[
   \begin{@twocolumnfalse}
     \printbibliography
   \end{@twocolumnfalse}
]

\end{document}
